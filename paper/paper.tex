\documentclass[12pt,a4paper]{article}
% vim: set textwidth=100:

\usepackage{amsmath}
\usepackage{amsfonts}
\usepackage{amsthm}
\usepackage{hyperref}

\title{Data Structures Comparative Review Thing}
\author{Pieter Brederode, Timo Koppenberg, \\ Jelco Bodewes, Anton Golov}

\begin{document}
    \maketitle

    \begin{abstract}
        We did research and got results.
    \end{abstract}


    \section{Introduction}
    
    We did research about what data structure could best be used to represent a blacklist in a 
    web-filter application.  We compared the insertion speed, query speed, and memory usage of three
    data structures: the binary search tree, the hash table, and the skip list.  We did this because
    we really liked these three structures. 

	\section{Implementation}
	
	We used C++ as the implementation language.  libstdc++'s implementations of the standard
	class templates \texttt{std::map} and \texttt{std::unordered\_map} were used for the binary search
	tree and the hash table respectively.  As the C++ standard library lacks a skip list\footnote{reference to
	the standard here}, we used an open-source third party library, \texttt{CSSkipList}\footnote{reference to
	library home page here}.
	
	For our measurements, we used libstdc++'s implementation of the standard
	\texttt{std::chrono::high\_resolution\_clock} class was used for time measurements.   Memory
	measurements were performed by providing a custom allocator which tracked the number of memory
	requests.  Note that the latter is a step away from our initial plan, where we stated we would use
	the POSIX API\footnote{ref to the posix api}; unfortunately, that turned out to be too coarse-grained
	for our purposes and we could thus get no accurate measurements that way.  The allocator-based technique
	we employed required slight modifications to the skip list implementation we used, but this
	shouldn't have caused a significant slowdown (if anything, it made things more fair, as now allocation
	was done the same way in both systems).
	

    \section{Results}

	Our results were great!  They are best analysed with a two-sample independent T test.  We had an
	equal number of measurements about group 1 (with mean $x_1$ and standard deviation $\sigma_1$ and
	about group 2 (with mean $x_2$ and standard deviation $\sigma_2$).  From this we can conclude that
	data structure 1 is, on average, unquestionably superior.


    \section{Conclusions}

    The skip list uses the least memory but is awful in every other imaginable way.  The hashmap is
    usually the best, except when we're dealing with a small number of integer keys and doing far more
    inserts than queries.  Note that this doesn't actually happen in practice.

    \bibliographystyle{alpha}

    \bibliography{paper}

\end{document}
